\documentclass[naustrian]{scrartcl}
\usepackage{cook}

\usepackage[T1]{fontenc}
\usepackage{wallpaper}

\newcommand\kochbuchauthor{Sebastian Preisner & Nicole Neumann}
\newcommand\kochbuchurl{http://www.calyrium.org/}
%\newcommand\kochbuchurl{\href{http://sourceforge.net/p/chilicookbook}{\nolinkurl{http://sourceforge.net/p/chilicookbook}}}
\newcommand\kochbuchtitle{Fitness Sammlung}
\newcommand\kochbuchversion{Ver. 0.1}

\newcommand{\caps}{www.Capsamania.de}

\setcounter{secnumdepth}{-1} % um die nummerierung von sections zu unterbinden und trotzdem die PDF-Bookmarks zu haben

% Schriftarten fuer Ueberschriften
\usepackage{aurical}
\usepackage{calligra}
\usepackage{yfonts}
\usepackage{uncial}
\usepackage{rustic}
%\usepackage{emerald}
%\usepackage{rotunda}
%\usepackage{inslrmin}
%\usepackage{pbsi}
%\usepackage{egothic}
%\usepackage{carolmin}
%%\usepackage{chancery}
%\usepackage{sqrcaps}


\definecolor{darkblue}{rgb}{0,0,.5}
\usepackage[ % muss letztes Package sein!
	pdftitle={\kochbuchtitle},%
	pdfauthor={\kochbuchauthor},%
	pdfsubject={Kochbuch},%
	pdfcreator={PDFLaTeX},
	colorlinks=true, urlcolor=darkblue, linkcolor=darkblue, bookmarksopen=true
 ]{hyperref} % 

%\twosided 	% Zweiseitig?

\usekomafont{section}
\addtokomafont{section}{\newpage\centering\vspace*{2cm}\rmfamily\Huge\color{darkred}}

% ============================= Einheiten ==============================
\newcommand{\g}{\,g }
\newcommand{\kg}{\,kg }
\newcommand{\ml}{\,ml }
\renewcommand{\l}{\,l }
\newcommand{\TL}{\,TL }
\newcommand{\EL}{\,EL }
\newcommand{\cm}{\,cm }
\newcommand{\m}{\,m }
\renewcommand{\min}{\,min }
\newcommand{\pkg}{\,Päckchen}

% ============================= Abkuerzungen ==============================
\newcommand{\bzw}{bzw.\@\xspace}
\newcommand{\bzgl}{bzgl.\@\xspace}
\newcommand{\ca}{ca.\@\xspace}
\newcommand{\dah}{d.\thinspace{}h.\@\xspace}
\newcommand{\Dah}{D.\thinspace{}h.\@\xspace}
\newcommand{\ds}{d.\thinspace{}sind\@\xspace}
\newcommand{\evtl}{evtl.\@\xspace}
\newcommand{\ua}{u.\thinspace{}a.\@\xspace}
\newcommand{\Ua}{U.\thinspace{}a.\@\xspace}
\newcommand{\usw}{usw.\@\xspace}
\newcommand{\va}{vor allem\@\xspace}
\newcommand{\vgl}{vgl.\@\xspace}
\newcommand{\zB}{z.\thinspace{}B.\@\xspace}
\newcommand{\ZB}{Zum Beispiel\xspace}
\newcommand{\sa}{s.\ auch\@\xspace}
\newcommand{\ia}{i.\thinspace{}Allg.\@\xspace}
\newcommand{\su}{s.\ unten\@\xspace}
\newcommand{\uvm}{u.\thinspace{}v.\thinspace{}m.\@\xspace}
\newcommand{\uva}{u.\thinspace{}v.\thinspace{}a.\@\xspace}
\newcommand{\uae}{u.\thinspace{}ä.\@\xspace}

\begin{document}

% entweder die Optionen ausfüllen und mit \maketitle Titelseite erzeugen oder...
%\title{Meine Küche}
%\author{Autor}

% ... die Titelseite selber zusammenstellen
\pdfbookmark[1]{Titel}{Meine Fitnessküche}
\begin{titlepage}
		\begin{center}
		\textcolor{white}{\\[6.5cm]\bfseries{	\scalebox{5.5}{Fitness} \\[1,5cm] \scalebox{5.5}{Küche} 
				\\[12cm]	\scalebox{1.8}{\kochbuchurl}} \\ \centering\vspace*{1cm}\copyright \ 2015 \kochbuchauthor \ \ \kochbuchversion}
		\end{center}
		%\ThisCenterWallPaper{1.52}{./bilder/start2.jpg}
%		\begin{figure}[H]% einfügen der Grafik
%			\centering\includegraphics[width=0.95\textwidth]{./bilder/start.jpg}
%			%\centering\includegraphics[height=0.94\textheight]{./bilder/start.jpg}
%		\end{figure}
%\newpage
%\centering\vspace*{3cm}\copyright \ 2011 \kochbuchauthor
\end{titlepage}

\pdfbookmark[1]{Inhaltsverzeichnis}{toc}
\tableofcontents

% Standardfarbe, -schriftart für die Überschrift
\recipecolor{EFEFEF}

% Schriftart der Ueberschrift
%\recipefont{\rustfamily}
%\recipefont{\calligra}
%\recipefont{\Fontskrivan}
\recipefont{\Fontlukas}
%\recipefont{\Fontamici}
%\recipefont{\cminfamily}
%\recipefont{\rmfamily}
%\recipefont{\ECFJD}

% wenn \twosided aktiv ist (bei Zweiseitig ist die ungerade Zahl immer auf der Vorderseite)
% müssen hier soviele newpages einfügen bis die erste Sektion auf einer ungeraden Seite beginnt
\newpage~
\subsection*{\centering Danksagung}
\begin{quote}
Vielen Dank an \emph{Stefan Gabauer} der mit seinem Chili Cook Book\footnote{\url{http://sourceforge.net/projects/chilicookbook}} die \LaTeX-Stil-Vorlage von \emph{Christian Gatzlaff} zugänglich gemacht hat und uns somit das Werkzeug zur Erstellung dieses Fitnesskochbuchs in die Hand gegeben hat.\\

Wir möchten uns natürlich bei all den Kreativen Köpfen bedanken die ihre Rezeptideen mit uns geteilt haben und uns so zu dieser Zusammenstellung Inspiriert haben.
\vspace{2cm}

\subsection*{\centering Lizenz}
\begin{center}
"`Fitness Küche"'\\
\textit{Dieses Kochbuch zeigt das gesunde Ernährung auch gut schmecken kann!}\\
Copyright (C) 2015 Sebastian Preisner\\

\begin{center}
	\href{http://creativecommons.org/licenses/by-nc-sa/3.0/de/}{\includegraphics{cclogo.png}}
\end{center}

Fitness Küche von Sebastian Preisner & Nicole Neumann steht unter einer \\\href{http://creativecommons.org/licenses/by/4.0/}{Creative Commons Namensnennung 4.0 International Lizenz.}\\
\url{http://creativecommons.org/licenses/by/4.0/}
\end{center}
%%%%%%%%%%%%%%%%%%%%%%%%%%%%%%%%%%%%%%%%%%%%%%%%%%%%%%%%%%%%%%%%%%%%%%%%%%%%%%%%%%%%%%%%%%%%%%%


\section{Konservieren} %%%%%%%%%%%%%%%%%%%%%%%%%%%%%%%%% Konservieren

\section{Frühstück} %%%%%%%%%%%%%%%%%%%%%%%%%%%%%%%%% Konservieren


\section{Suppen} %%%%%%%%%%%%%%%%%%%%%%%%%%%%%%%% Suppen


\section{Hauptgerichte} %%%%%%%%%%%%%%%%%%%%%%%%%%%%%%%% Hauptgerichte
\input{./rezepte/Thunfischfrikadellen_mit_Asia-salat.tex}
%\input{./rezepte/Chilidelfruttidiavolo.tex}
%\input{./rezepte/GebrateneChili-NudelnmitEiundPilzen.tex}


\section{Saucen} %%%%%%%%%%%%%%%%%%%%%%%%%%%%%%%%%%%%%%% Soucen
%\input{./rezepte/Sambal.tex}
%\input{./rezepte/Chilichutney.tex}
%\input{./rezepte/ChileconQueso.tex}
%\input{./rezepte/Habanero-Ingwer-Nektarinen-Erdbeer-Sauce.tex}
%\input{./rezepte/Limetten-Minz-Hot-Sauce.tex}
%\input{./rezepte/LemonDrop-AnanasHotSauce.tex}
%\input{./rezepte/Suess-scharfeThaisauce.tex}
%\input{./rezepte/Chili-Apfel-Paste.tex}

\section{Salate} %%%%%%%%%%%%%%%%%%%%%%%%%%%%%%%% Salate
%\input{./rezepte/BrombeerenGarnelenSalat.tex}

\section{Snaks} %%%%%%%%%%%%%%%%%%%%%%%%%%%%%%%% f�r Zwischendurch
%\input{./rezepte/SchnelleHabisandwiches.tex}

\section{Süßes} %%%%%%%%%%%%%%%%%%%%%%%%%%%%%%%%%%%%%%%% Suesses


\section{Medizinisches} %%%%%%%%%%%%%%%%%%%%%%%%%%%%%%%%%%%%%%%% Medizinisches


\section{Infos und Wissenswertes} %%%%%%%%%%%%%%%%%%%%%%%%%%%%%%%%%%%%%%%%% Info
%\input{./Info.tex}
	
\end{itemize}

%%%%%%%%%%%%%%%%%%%%%%%%%%%%%%%%%%%%%%%%%%%%%%%%%%%%%%%%%%%%%%%%%%%%%%%%%%%%%%%%%%%%%%%%%%%%%%%
\end{document}