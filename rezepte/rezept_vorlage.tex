% ====== Rezeptname und die Quelle ======
\begin{recipe}[]{ REZEPTNAME }{ QUELLE bzw Author des Rezeptes}{ QUELLE des Bildes }

% ====== Zeit, Personen und Schärfe ======
%\timerecipe{ca. 1}         % Zubereitungszeit in Stunden
\timerecipe[Minuten]{10}    % oder in Minuten
\personcount{4}        		% Personenanzahl
\spicecount{3}              % Schaerfe von 5

% ====== Zutaten ======
\ingredient{2 frische Chilis}
\ingredient{2 \EL Ketchup}

         % ====== Zubereitung ======    
\step
hier den zubereitungsschritt beschreiben

\step
hier den zubereitungsschritt beschreiben

\tippbox{Dies ist ein Hinweis um noch ein paar Zusatzinformationen und weitere Anregungen dem Rezept beizufügen.}

\step
hier den zubereitungsschritt schreiben

\tippboxtip{Dies ist ein umrahmter Tipp um noch ein paar Zusatzinformationen und weitere Anregungen dem Rezept beizufügen.} 
% Tipp in extra Rahmen mit dem Wort "Tipp:" am Anfang

         % ====== Bild ======
% Grafik fuer das Rezept koennen so eingefuegt werden:
% wenn kein Bild vorhanden ist, bitte diese Zeile auskommentiert lassen.
%\graphic{./bilder/todo.jpg}
\end{recipe}