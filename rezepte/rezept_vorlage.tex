% ====== Rezeptname und die Quelle ======
\begin{recipe}[]{ Titel }{ Quelle und/oder Autor des Rezepts }{ Quelle des Bildes }

% ====== Zeit, Personen und Schärfe ======
%\timerecipe{ca. 1}         % Zubereitungszeit in Stunden
\timerecipe[Minuten]{10}    % oder in Minuten
\personcount{4}        		% Personenanzahl
\spicecount{3}              % Schaerfe von 5

% ====== Zutaten ======
\ingredient{3 Eier}
\ingredient{\sfrag{1}{4} Tasse geschmolzene Butter}
\ingredient{\sfrag{1}{4} Tasse Zuckeraustauschstoff (nach belieben)}
\ingredient{\sfraq{1}{3} Tasse Kokosnussmehl}
\ingredient{\sfraq{1}{2} Tasse Himbeeren, Blaubeeren oder Beerenmix}
\ingredient{\sfraq{1}{2} \TL Backpulver}
\ingredient{\sfraq{1}{2} \TL Vanille}
\ingredient{1 Prise Salz}
\ingredient{2-5 \EL Wasser}

% ====== Zubereitung ======    
\step
Den Backofen auf 190 Grad vorheizen. Währenddessen die Eier verquierlen. Nach und nach die geschmolzene Butter, die Vanille, das Salz sowie den Zuckeraustauschstoff zu den Eiern hinzugeben. 

\step
hier den zubereitungsschritt beschreiben

\tippbox{Dies ist ein Hinweis um noch ein paar Zusatzinformationen und weitere Anregungen dem Rezept beizufügen.}

\step
hier den zubereitungsschritt schreiben

% Tipp in extra Rahmen mit dem Wort "Tipp:" am Anfang
\tippboxtip{Dies ist ein umrahmter Tipp um noch ein paar Zusatzinformationen und weitere Anregungen dem Rezept beizufügen.} 

         % ====== Bild ======
% Grafik fuer das Rezept koennen so eingefuegt werden:
% wenn kein Bild vorhanden ist, bitte diese Zeile auskommentiert lassen.
%\graphic{./bilder/todo.jpg}
\end{recipe}