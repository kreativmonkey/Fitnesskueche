% ====== Rezeptname und die Quelle ======
\begin{recipe}[]{ Afterworkout-Beerenmix }{}{}

Dieser Shake zeichnet sich durch den hohen Eiweißgehalt aus und dient so als potentielle Eiweißquelle für nach dem Training, aber auch zur Erfrischung an Warmen Tagen kann der Trink gemacht werden.
% ====== Zeit, Personen und Schärfe ======
%\timerecipe{ca. 1}         % Zubereitungszeit in Stunden
\timerecipe[Minuten]{5}    % oder in Minuten
\personcount{1}        		% Personenanzahl
\spicecount{0}              % Schaerfe von 5

% ====== Zutaten ======
\ingredient{150\g \tk Beerenmix}
\ingredient{500\g Magerquark}
\ingredient{150\ml Wasser}
\ingredient{1 Vanilleschote}


         % ====== Zubereitung ======    
\step
Die Beeren zusammen mit dem Magerquark und dem Wasser in einem Mixer bei höchster Stufe \ca 1\min mixen.

\step
Die Vanilleschote mit einem Messer vorsichtig einschneiden und öffnen. Mit dem Messerrücken die Schote auskratzen und den Inhalt in den Mixer geben. Alles kurz durchmischen und fertig ist der Shake.

\tippboxtip{Dieses Rezept ist als Grundlage zu verstehen. Durch das verändern der Zutaten entstehen immer neue Geschmacksrichtungen. Probiere es aus. } 
% Tipp in extra Rahmen mit dem Wort "Tipp:" am Anfang

         % ====== Bild ======
% Grafik fuer das Rezept koennen so eingefuegt werden:
% wenn kein Bild vorhanden ist, bitte diese Zeile auskommentiert lassen.
%\graphic{./bilder/todo.jpg}
\end{recipe}