% ====== Rezeptname und die Quelle ======
\begin{recipe}[]{ Gebackene Süßkartoffeln mit Avocado-Paprika-Dip }{ abwandlung von \href{http://www.chefkoch.de/rezepte/2725361425059942/Gebackene-Suesskartoffeln-mit-Avocado-Paprika-Creme.html}{cschoenbrodt} auf Chefkoch }

% ====== Zeit, Personen und Schärfe ======
%\timerecipe{ca. 1}         % Zubereitungszeit in Stunden
\timerecipe[Minuten]{45}    % oder in Minuten
\personcount{2}        		% Personenanzahl
\spicecount{1}              % Schaerfe von 5

% ====== Zutaten ======
\ingredient{3-4 ca. 1\kg Süßkartoffeln}
\ingredient{3 Avocado}
\ingredient{200\g Magerquark}
\ingredient{200\g Hüttenkäse}
\ingredient{3-4 Paprikaschoten (bunt)}
\ingredient{4\EL Mandeln, gehackt}
\ingredient{1 Zwiebel}
\ingredient{1 Knoblauchzehe}
\ingredient{1 Bund Petersilie}
\ingredient{2\EL Limettensaft}
\ingredient{1\TL Chilliflocken oder -ringe}
\ingredient{4\TL Raps- oder Olivenöl}
\ingredient{Salz (grob), Pfeffer (grob)}
  	
         % ====== Zubereitung ======    
\step
Süßkartoffeln gründlich waschen und nebeneinander auf ein Backblech legen. 
Öl mit grobem Salz und geschrotetem Pfeffer vermischen und die Süßkartoffeln rundherum damit bepinseln. 
Im Ofen bei etwa 200°C je nach Größe ca. 50 Minuten garen.

\step
Zwiebel und Knoblauchzehe abziehen und fein würfeln. Öl in einer Pfanne erhitzen, Knoblauch und Zwiebel darin bräunen.
Die Avocados halbieren, entkernen und das Fruchtfleisch herauslösen. Dieses mit Magerquark, Limettensaft, 
Hüttenkäse, Zwiebel und Knoblauch vermischen.

\step
Paprikaschoten ganz fein würfeln, Petersilie fein hacken. \sfrac{3}{4} der Petersilie und der gewürfelten Paprika zur Masse geben. Alles mit Salz, Pfeffer und Chiliflocken abschmecken.

\step
Die gegarten Süßkartoffeln der Länge nach einschneiden und mit dem Avocado-Paprika-Dip auf einem Tellern anrichten. Mit gehackten Mandeln, Chiliflocken, Petersilie und den verbliebenen Paprikawürfeln bestreuen und servieren. 


% Tipp in extra Rahmen mit dem Wort "Tipp:" am Anfang
% \tippboxtip{Dies ist ein umrahmter Tipp um noch ein paar Zusatzinformationen und weitere Anregungen dem Rezept beizufügen.} 

         % ====== Bild ======
% Grafik fuer das Rezept koennen so eingefuegt werden:
% wenn kein Bild vorhanden ist, bitte diese Zeile auskommentiert lassen.
% \graphic{./bilder/Putenstreifen_mit_Erdnusschilisauce.jpg}
\end{recipe}
