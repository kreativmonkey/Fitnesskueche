% ====== Rezeptname und die Quelle ======
\begin{recipe}[]{ Lachsfilet mit Spinat und Avocadosalat }{ \href{www.fitnessmagnet.com/Rezepte/tabid/80/Fitness/View/Fitnessartikel/1637/Lachsfilet-an-einem-Avocado-Dip-mit-frischem-Spinat.aspx} }{  }

% ====== Zeit, Personen und Schärfe ======
%\timerecipe{ca. }         % Zubereitungszeit in Stunden
\timerecipe[Minuten]{30}    % oder in Minuten
\personcount{2}        		% Personenanzahl
\spicecount{1}              % Schaerfe von 5

% ====== Zutaten ======
\ingredient{}
\textbf{Lachsfilet}
\ingredient{}
\ingredient{300-400\g Lachsfilet}
\ingredient{4 Zweige Dill}
\textbf{Spinat:}
\ingredient{}
\ingredient{2\EL Ingwer-Honig-Essig (o.ä.)}
\ingredient{etwas Speckwürfel}
\ingredient{30\g Pinienkerne}
\ingredient{frischen Blattspinat}
\ingredient{etwas Parmesan}
\ingredient{4-6 Cherry Tomaten}
\ingredient{Salz}
\textbf{Avocado Salat:}
\ingredient{}
\ingredient{1 große Avocado}
\ingredient{1 rote Zwiebel}
\ingredient{1 Limette o. Zitrone}
\ingredient{\sfrac{1}/{2} Chilli o. Chilliflocken}
\ingredient{2\EL Olivenöl}
\ingredient{1 kl. Dose Mais}
\ingredient{5 Blätter Basilikum}
\ingredient{salz/Pfeffer}

% ====== Zubereitung ======    
\step
Ofen auf 200C° Ober-/Unterhitze Vorheitzen. Die Avocado halbieren, entkernen, aushöhlen und 
das Fleisch in einer Schüssel mit der Gabel zerdrücken. 
Die Zwiebel würfeln, den Basilikum zerkleinern die Limette halbieren und den Saft auspressen. 
Die Zutaten mit der entkernten, geschnittenen Chilli zur Avocado hinzugeben und gut vermischen.
Die ganze Masse mit Salz und Pfeffer abschmecken und in einer Schale anrichtzen.

\step
Das Lachfilet in 4 gleichgroße teile Aufteilen und mit Olivenöl beträufeln. 
Zusammen mit den Dillzweigen in den Backofen ca. 15 Minuten auf mittlerer Schiene garen.

\tippboxvar{Der Lach kann wahlweise auch in der Pfannen angebraten oder gegrillt werden}

\step
In einem Topf die Speckwürfel und die Pinienkernen anbraten. Danach mit dem Essig ablöschen, die Spinatblätter hinzugeben und mit ein wenig Salz würzen.
\tippbox{der Spinat soll bloss warm werden, denn nach ca. 30 Sekunden fällt er schon zusammen!}

\step
Den Spinat zusammen mit dem Speck und den Pinienkernen herausnehmen, auf einem Teller mit dem Lachfilets
und dem Dill anrichten und mit Parmesan und geschnittenen Cherry-Tomaten garnieren.

\tippboxtip{Das Rezept funktioniert auch hervorragend mit Tief kühl Spinat. für 2 Peronen sollten hier ca. 400\g TK Spinat verwendet werden}

         % ====== Bild ======
% Grafik fuer das Rezept koennen so eingefuegt werden:
% wenn kein Bild vorhanden ist, bitte diese Zeile auskommentiert lassen.
%\graphic{./bilder/todo.jpg}
\end{recipe}
