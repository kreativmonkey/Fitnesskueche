% ====== Rezeptname und die Quelle ======
\begin{recipe}[]{ Falscher Reis }{ Chefkoch: http://www.chefkoch.de/rezepte/2506181393345016/Falscher-Reis.html }{  }

Falscher Reis besteht aus Blumenkohl und zeichnet sich durch seinen geringen Kohlenhydratanteil aus. Er eignet sich gut als Beilage und kann wie Reis verwendet werden.

% ====== Zeit, Personen und Schärfe ======
%\timerecipe{ca. 1}         % Zubereitungszeit in Stunden
\timerecipe[Minuten]{15}    % oder in Minuten
\personcount{2}        		% Personenanzahl
\spicecount{0}              % Schaerfe von 5

% ====== Zutaten ======
\ingredient{700\g Blumenkohl}
\ingredient{2\EL Butter}
\ingredient{1 Prise Salz}
\ingredient{etwas Muskat}

% ====== Zubereitung ======    
\step
Den gewaschenen Blumenkohl in Röschen zerteilen und in einen Mixer geben. Vorsichtig auf höchster Stufe die Röschen zerkleinern, so das sie die Größe von Reiskörner bekommen. Darauf achten das es nicht mußig wird. 

\step
Die Blumenkohlreiskörner in ein Mikrowellen geeignetes Gefäß geben und abgedeckt etwa 5-8 Minuten (je nach Leistung) in der Mikrowelle garen. Nach der Hälfte einmal kurz durchschütteln.

\step
Nach dem Garen mit Butter, Salz und Muskatnuss abschmecken.

% Tipp in extra Rahmen mit dem Wort "Tipp:" am Anfang
\tippboxtip{Passt besonders gut zu Gerichten mit viel Soße.} 

         % ====== Bild ======
% Grafik fuer das Rezept koennen so eingefuegt werden:
% wenn kein Bild vorhanden ist, bitte diese Zeile auskommentiert lassen.
%\graphic{./bilder/todo.jpg}
\end{recipe}