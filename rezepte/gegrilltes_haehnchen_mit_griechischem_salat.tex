% ====== Rezeptname und die Quelle ======
\begin{recipe}[]{  Gegrilltes Hähnchen mit griechischem Salat  }{ \href{http://www.fitnessmagnet.com/Rezepte/tabid/80/Fitness/View/Fitnessartikel/1829/Gegrilltes-Hahnchen-mit-griechischem-Salat.aspx}{fitnessmagnet.com} }{  }

% ====== Zeit, Personen und Schärfe ======
%\timerecipe{ca. 4}         % Zubereitungszeit in Stunden
\timerecipe[Minuten]{15}    % oder in Minuten
\personcount{2}        		% Personenanzahl
\spicecount{0}              % Schaerfe von 5

% ====== Zutaten ======
\ingredient{200\g - 400\g Hähnchenbrust}
\ingredient{2 mittelgroße Tomaten}
\ingredient{\sfrac{1}{2} Gurke}
\ingredient{1 rote Zwiebel}
\ingredient{10 Oliven}
\ingredient{100\g Feta}
\ingredient{1\EL Olivenöl}
\ingredient{Essig nach belieben}
\ingredient{Meersalz, Pfeffer}
\ingredient{Knoblauchpulver}
\ingredient{\getr Oregano}

% ====== Zubereitung ======
\step
Hähnchen nach Belieben würzen und mit etwas Olivenöl scharf anbraten oder grillen.

\step
Die Tomaten, Oliven, Gurke und den Feta Würfeln. Die Zwiebel fein Hacken. Alles in einer Schüssel gut vermischen.
Mit Essig, Meersalz, und Kräutern abschmecken.

\step
Das Hähnchen auf einen Teller anrichten und mit dem Salat garnieren. Guten Appetit!

% \tippbox{Dies ist ein Hinweis um noch ein paar Zusatzinformationen und weitere Anregungen dem Rezept beizufügen.}

% Tipp in extra Rahmen mit dem Wort "Tipp:" am Anfang
% \tippboxtip{Dies ist ein umrahmter Tipp um noch ein paar Zusatzinformationen und weitere Anregungen dem Rezept beizufügen.}

         % ====== Bild ======
% Grafik fuer das Rezept koennen so eingefuegt werden:
% wenn kein Bild vorhanden ist, bitte diese Zeile auskommentiert lassen.
%\graphic{./bilder/todo.jpg}
\end{recipe}
