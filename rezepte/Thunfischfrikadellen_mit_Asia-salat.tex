% ====== Rezeptname und die Quelle ======
\begin{recipe}[]{ Thunfischfrikadellen mit Asia-Salat }{ Freeletics Ernährungsguid }

% ====== Zeit, Personen und Schärfe ======
%\timerecipe{ca. 1}         % Zubereitungszeit in Stunden
\timerecipe[Minuten]{15}    % oder in Minuten
\personcount{1}        		% Personenanzahl
\spicecount{0}              % Schaerfe von 5

% ====== Zutaten ======
\ingredient{}
\textbf{Frikadellen:}
\ingredient{}
\ingredient{150\g Thunfisch in der Dose}
\ingredient{30\g Haferflocken oder Hafermehl}
\ingredient{1 Ei}
\ingredient{\sfrac{1}{2} Zwiebel}
\textbf{Salat:}
\ingredient{}
\ingredient{1 Pak Choi (\ca 200\g)}
\ingredient{1 Frühlingszwiebel}
\ingredient{100\g Soja- oder Mungobohnensprossen}
\ingredient{2 \EL Sesam (\ca 20\g)}
\ingredient{1 daumengroßes Stück Ingwer}
\ingredient{\sfrac{1}{2} Limette}
\ingredient{3 \EL Sojasauce}

         % ====== Zubereitung ======    
\step
Den Thunfisch mit einer Gabel zerpflücken. Haferflocken in einem Mixer auf höchster Stufe zu einer Art Mehl verarbeiten. Zwiebeln abziehen und klein schneiden. Den Sesam währenddessen in einer Pfanne ohne Fett auf mittlerer Hitze \ca 3 Minuten anrösten. 

\step
Thunfisch, Haferflocken, Ei und Zwiebelwürfel gründlich vermengen. Aus der so entstehenden Masse werden nun 4 gleichgroße Frikadellen geformt. 

\step 
Öl in einer Pfanne bei mittlerer Hitze heiß werden lassen und die Thunfischfrikadellen von jeder Seite \ca 4 Minuten braten.\\ 
Währenddessen Pak Choi, Paprika, Frühlingszwiebel und -falls nötig- Sprossen waschen, trocken tupfen.

\tippbox{Frische Sprossen schmecken einfach am besten. Gibt es oft in großen Supermärkten zu erwerben oder beim Asialaden um die Ecke.}

\step
Pak Choi und Paprika in Streifen schneiden, Frühlingszwiebel in Ringe schneiden und mit den Sprossen und dem Sesam mischen. 

\step 
Den Ingwer nach dem schälen auf einer Reibe fein raspeln und zusammen mit dem Saft der ausgepressten Limette und der Sojasauce auf den Salat geben.

\tippboxtip{Dies ist ein umrahmter Tipp um noch ein paar Zusatzinformationen und weitere Anregungen dem Rezept beizufügen.} 
% Tipp in extra Rahmen mit dem Wort "Tipp:" am Anfang

         % ====== Bild ======
% Grafik fuer das Rezept koennen so eingefuegt werden:
% wenn kein Bild vorhanden ist, bitte diese Zeile auskommentiert lassen.
%\graphic{./bilder/todo.jpg}
\end{recipe}
