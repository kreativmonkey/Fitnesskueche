% ====== Rezeptname und die Quelle ======
\begin{recipe}[]{ Putenbruststreifen mit Erdnusschilisauce }{ Freeletics Nutrationguid }

% ====== Zeit, Personen und Schärfe ======
%\timerecipe{ca. 1}         % Zubereitungszeit in Stunden
\timerecipe[Minuten]{35}    % oder in Minuten
\personcount{2}        		% Personenanzahl
\spicecount{1}              % Schaerfe von 5

% ====== Zutaten ======
\ingredient{400\g Puten- oder Hänchenbrustfilet}
\ingredient{300\g Blumenkohl}
\ingredient{100\g Lauch}
\ingredient{4 Möhren}
\ingredient{200\ml Kokosmilch (ungesüßt)}
\ingredient{8\EL Erdnussbutter (ungesüßt)}
\ingredient{2 Knoblauchzehe}
\ingredient{2 daumengroße Stück Ingwer (40\g)}
\ingredient{1 Limette}
\ingredient{1\TL Chilliflocken oder -ringe}
\ingredient{4\TL Raps- oder Olivenöl}
\ingredient{Salz, Pfeffer}

         % ====== Zubereitung ======    
\step
Puten- oder Hänchenbrustfilet mit kalten Wasser abbrausen, trockentupfen und in \ca 1\cm dicke Streifen schneiden. Den Blumenkohl in Scheiben schneiden. Den Lauch waschen, trockentupfen und in Scheiben schneiden. Möhren schälen und in Scheiben schneiden.
 
\step
In einer Pfanne auf mittlerer Hitze 2\TL Olivenöl erhitzen und darin die Filetstreifen und das Gemüse \ca 2 Minuten von jeder Seite anbraten. Das Fleisch herausnehmen und beiseite legen. 150\ml Wasser auf das Gemüse geben und köcheln lassen bis das Wasser verdampft ist (\ca 5 \min)
\tippbox{Das Fleisch auf einem im Backofen vorgeheizten Teller bei 50 Grad in den Backofen.}

\step
Für die Sauce den Knoblauch und Ingwer schälen und in dünne Scheiben schneiden. Die Zitrone/Limette halbieren und auspressen. In einem Topf bei mittlerer Hitze 1\TL Olivenöl heiß werden lassen und den Knoblauch, Ingwer und die Cihili \ca 3 \min anbraten.

\step
Die Sauce mit Kokosmilch ablöschen, kurz aufkochen lassen und dann vom Herd nehmen. Nun wird die Erdnusscreme und der Limettensaft eingerührt, danach mit Salz und Pfeffer abschmecken.

\step
Putenstreifen mit Gemüse und Erdnusssauce anrichten und  genießen.

% Tipp in extra Rahmen mit dem Wort "Tipp:" am Anfang
\tippboxtip{Dies ist ein umrahmter Tipp um noch ein paar Zusatzinformationen und weitere Anregungen dem Rezept beizufügen.} 

         % ====== Bild ======
% Grafik fuer das Rezept koennen so eingefuegt werden:
% wenn kein Bild vorhanden ist, bitte diese Zeile auskommentiert lassen.
\graphic{./bilder/Putenstreifen_mit_Erdnusschilisauce.jpg}
\end{recipe}