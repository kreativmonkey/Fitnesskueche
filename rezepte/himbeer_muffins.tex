% ====== Rezeptname und die Quelle ======
\begin{recipe}[]{ Himbeer-Muffins }{\href{http://www.lowcarb-ernaehrung.info/rezept-kokos-himbeer-muffins/}{http://www.lowcarb-ernaehrung.info/rezept-kokos-himbeer-muffins/}}{ }

% ====== Zeit, Personen und Schärfe ======
%\timerecipe{ca. 1}         % Zubereitungszeit in Stunden
\timerecipe[Minuten]{10}    % oder in Minuten
\personcount{4}        		% Personenanzahl
\spicecount{0}              % Schaerfe von 5

% ====== Zutaten ======
\ingredient{3 Eier}
\ingredient{\sfrac{1}{4} Tasse geschmolzene Butter}
\ingredient{\sfrac{1}{4} Tasse Zuckeraustauschstoff (nach belieben)}
\ingredient{\sfrac{1}{3} Tasse Kokosnussmehl}
\ingredient{\sfrac{1}{2} Tasse Himbeeren, Blaubeeren oder Beerenmix}
\ingredient{\sfrac{1}{2}\TL Backpulver}
\ingredient{\sfrac{1}{2}\TL Vanille}
\ingredient{1\pr Salz}
\ingredient{2-5\EL Wasser}

% ====== Zubereitung ======
\step
Den Backofen auf 190 Grad vorheizen. Währenddessen die Eier verquierlen. Nach und nach die geschmolzene Butter, die Vanille, das Salz sowie den Zuckeraustauschstoff zu den Eiern hinzugeben. Anschließend das Kokosnussmehl sowie Backpulver zufügen. Den Teig so lange verrühren bis eine glatte Masse entsteht.

\tippbox{Je nach Konsistenz des Teiges etwas Wasser beimischen. Der Teig sollte nicht zu dick aber auch nicht zu dünn sein.}

\step
Die Beeren unter den Teig heben und dann in die Muffinformen. Die Muffins für 15 bis 20 Minuten in den Backofen geben bis sie gold-braun sind.

% Tipp in extra Rahmen und mit Hintergrundfarbe sowie dem Wort "Tipp:" am Anfang
%\tippboxtip{Dies ist ein umrahmter Tipp um noch ein paar Zusatzinformationen und weitere Anregungen dem Rezept beizufügen.}

         % ====== Bild ======
% Grafik fuer das Rezept koennen so eingefuegt werden:
% wenn kein Bild vorhanden ist, bitte diese Zeile auskommentiert lassen.
%\graphic{./bilder/todo.jpg}
\end{recipe}
