% ====== Rezeptname und die Quelle ======
\begin{recipe}[]{ Veggie Burger }{ lafauxnoire }{  }

% ====== Zeit, Personen und Schärfe ======
\timerecipe{ca. 40}         % Zubereitungszeit in Stunden
%\timerecipe[Minuten]{10}    % oder in Minuten
\personcount{4}        		% Personenanzahl
\spicecount{0}              % Schaerfe von 5

% ====== Zutaten ======
\ingredient{150\g Champignons}
\ingredient{425\g rote Kidneybohnen}
\ingredient{1 Knoblauchzehe}
\ingredient{1 Zwiebel}
\ingredient{1\TL gemahlener Koriander}
\ingredient{1\TL gemahlener Kreutzkümmel}
\ingredient{2\EL glatte Petersilie, frisch gehackt}
\ingredient{1\EL Oliven-/Rapsöl}
\ingredient{Salz, Pfeffer}
\ingredient{Mehl}
\ingredient{1 Ei}

% ====== Zubereitung ======    
\step
Knoplauch, Zwiebel und Champignons fein hacken. In einer Pfanne das Öl erhitzen und die Zwiebeln für 5 Minuten weich dünsten. Danach den Koriander, Kreuzkümmel und den Knoblauch zugeben und unter gelegentlichem Rühren 1 Minute dünsten. Nun die Pilze zugeben und für 5 Minuten garen.

\step
Währenddessen die Bohnen abtropfen und abspülen und danach in einer Schüssel mit einem Kartoffelstampfer oder einer Gabel zerdrücken. Die Petersilie hinzugeben und verrühren.

\tippbox{Gib an dieser Stelle ein Ei hinzu damit die Masse besser zusammenhält. Es geht jedoch auch ohne Ei}

\step
Die Pilzmischung zu den Bohnen hinzugeben und gut Mischen. 

\step 
Die Masse in 4 Portionen aufteilen und in etwas Mehl wenden. Die Teile zu flachen, runden Burgern formen und mit Öl bestreichen.

\step 
Die Taler auf einem Blech verteilen und in den Vorgeheizten Backofen (Grill oder 200 Grad Oberhitze) für 5-7 Minuten von jeder Seite grillen.

\tippbox{ Funktioniert auch in der Pfanne. }

% Tipp in extra Rahmen mit dem Wort "Tipp:" am Anfang
\tippboxtip{Variation: Die Pilze lassen sich auch Prima durch eine Zucchini-Karotten Mischung ersetzen.} 

         % ====== Bild ======
% Grafik fuer das Rezept koennen so eingefuegt werden:
% wenn kein Bild vorhanden ist, bitte diese Zeile auskommentiert lassen.
%\graphic{./bilder/todo.jpg}
\end{recipe}