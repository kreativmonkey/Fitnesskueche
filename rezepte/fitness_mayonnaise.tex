% ====== Rezeptname und die Quelle ======
\begin{recipe}[]{ Fitness Mayonnaise }{ http://eatsmarter.de/rezepte/leichte-mayonnaise }{  }

% ====== Zeit, Personen und Schärfe ======
\timerecipe{ca. 7}         % Zubereitungszeit in Stunden
%\timerecipe[Minuten]{10}    % oder in Minuten
\personcount{4}        		% Personenanzahl
\spicecount{0}              % Schaerfe von 5

% ====== Zutaten ======
\ingredient{200\g Magerquark}
\ingredient{4 Eier}
\ingredient{\sfrac{1}{2} Zitrone}
\ingredient{1\EL Weißweinessig}
\ingredient{1-2\EL Wasser}
\ingredient{1\msp Cayennepfeffer}
\ingredient{Salz}

% ====== Zubereitung ======    
\step
Den Quark in einem feinen Sieb ca. 4 Stunden abtropfen lassen.

\step
Die Eier in einem kleinen Topf mit Wasser bedecken und zum Kochen bringen. Genau 9 Minuten kochen und unter kaltem Wasser abschrecken. Danach werden die Eier gepellen und halbiert und das Eigelb mit einem Teelöffel herausgelöst.  

\tippbox{Eiweiße anderweitig verwenden, z. B. hacken und über einen Salat streuen.}

\step
Das Wasser zum Eigelb geben und mit einem Holzlöffel zu einer cremigen Masse zerdrücken. Den Cayennepfeffer hinzugeben und nach und nach den abgetropften Quark unterrühren.

\step
Die Zitrone auspressen und 1-2\TL vom Saft mit dem Essig unter die Quarkeimasse mischen. 

\step 
Die fertige Masse nun noch 2-3 Stunden kalt stellen und genießen.

% Tipp in extra Rahmen mit dem Wort "Tipp:" am Anfang
%\tippboxtip{ } 

         % ====== Bild ======
% Grafik fuer das Rezept koennen so eingefuegt werden:
% wenn kein Bild vorhanden ist, bitte diese Zeile auskommentiert lassen.
%\graphic{./bilder/todo.jpg}
\end{recipe}